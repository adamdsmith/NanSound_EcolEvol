\documentclass[12pt,]{article}
\usepackage{lmodern}
\usepackage{amssymb,amsmath}
\usepackage{ifxetex,ifluatex}
\usepackage{fixltx2e} % provides \textsubscript
\ifnum 0\ifxetex 1\fi\ifluatex 1\fi=0 % if pdftex
  \usepackage[T1]{fontenc}
  \usepackage[utf8]{inputenc}
\else % if luatex or xelatex
  \ifxetex
    \usepackage{mathspec}
    \usepackage{xltxtra,xunicode}
  \else
    \usepackage{fontspec}
  \fi
  \defaultfontfeatures{Mapping=tex-text,Scale=MatchLowercase}
  \newcommand{\euro}{€}
\fi
% use upquote if available, for straight quotes in verbatim environments
\IfFileExists{upquote.sty}{\usepackage{upquote}}{}
% use microtype if available
\IfFileExists{microtype.sty}{%
\usepackage{microtype}
\UseMicrotypeSet[protrusion]{basicmath} % disable protrusion for tt fonts
}{}
\usepackage[margin=1in]{geometry}
\usepackage{color}
\usepackage{fancyvrb}
\newcommand{\VerbBar}{|}
\newcommand{\VERB}{\Verb[commandchars=\\\{\}]}
\DefineVerbatimEnvironment{Highlighting}{Verbatim}{commandchars=\\\{\}}
% Add ',fontsize=\small' for more characters per line
\usepackage{framed}
\definecolor{shadecolor}{RGB}{248,248,248}
\newenvironment{Shaded}{\begin{snugshade}}{\end{snugshade}}
\newcommand{\KeywordTok}[1]{\textcolor[rgb]{0.13,0.29,0.53}{\textbf{{#1}}}}
\newcommand{\DataTypeTok}[1]{\textcolor[rgb]{0.13,0.29,0.53}{{#1}}}
\newcommand{\DecValTok}[1]{\textcolor[rgb]{0.00,0.00,0.81}{{#1}}}
\newcommand{\BaseNTok}[1]{\textcolor[rgb]{0.00,0.00,0.81}{{#1}}}
\newcommand{\FloatTok}[1]{\textcolor[rgb]{0.00,0.00,0.81}{{#1}}}
\newcommand{\CharTok}[1]{\textcolor[rgb]{0.31,0.60,0.02}{{#1}}}
\newcommand{\StringTok}[1]{\textcolor[rgb]{0.31,0.60,0.02}{{#1}}}
\newcommand{\CommentTok}[1]{\textcolor[rgb]{0.56,0.35,0.01}{\textit{{#1}}}}
\newcommand{\OtherTok}[1]{\textcolor[rgb]{0.56,0.35,0.01}{{#1}}}
\newcommand{\AlertTok}[1]{\textcolor[rgb]{0.94,0.16,0.16}{{#1}}}
\newcommand{\FunctionTok}[1]{\textcolor[rgb]{0.00,0.00,0.00}{{#1}}}
\newcommand{\RegionMarkerTok}[1]{{#1}}
\newcommand{\ErrorTok}[1]{\textbf{{#1}}}
\newcommand{\NormalTok}[1]{{#1}}
\ifxetex
  \usepackage[setpagesize=false, % page size defined by xetex
              unicode=false, % unicode breaks when used with xetex
              xetex]{hyperref}
\else
  \usepackage[unicode=true]{hyperref}
\fi
\hypersetup{breaklinks=true,
            bookmarks=true,
            pdfauthor={},
            pdftitle={},
            colorlinks=true,
            citecolor=blue,
            urlcolor=blue,
            linkcolor=magenta,
            pdfborder={0 0 0}}
\urlstyle{same}  % don't use monospace font for urls
\setlength{\parindent}{0pt}
\setlength{\parskip}{6pt plus 2pt minus 1pt}
\setlength{\emergencystretch}{3em}  % prevent overfull lines
\setcounter{secnumdepth}{0}

%%% Use protect on footnotes to avoid problems with footnotes in titles
\let\rmarkdownfootnote\footnote%
\def\footnote{\protect\rmarkdownfootnote}

%%% Change title format to be more compact
\usepackage{titling}

% Create subtitle command for use in maketitle
\newcommand{\subtitle}[1]{
  \posttitle{
    \begin{center}\large#1\end{center}
    }
}

\setlength{\droptitle}{-2em}
  \title{}
  \pretitle{\vspace{\droptitle}}
  \posttitle{}
  \author{}
  \preauthor{}\postauthor{}
  \date{}
  \predate{}\postdate{}

\usepackage{animate}


\begin{document}

\maketitle


\section{Appendix 9. Seasonal animation of predicted scoter occupancy
and
abundance}\label{appendix-9.-seasonal-animation-of-predicted-scoter-occupancy-and-abundance}

To provide an example of the seasonal dynamics that can characterize sea
duck occupancy and abundance in Nantucket Sound, we proovide an
animation of predicted scoter occupancy and average abundance for every
other day between 1 November 2005 and 31 March 2006.

The code associated with this appendix
(\texttt{Appendix\_9\_(seasonal\_dynamics\_animation).Rmd}) produces two
files which contain the final animations:
\texttt{SCOT\_2005\_occup\_animation.pdf} and
\texttt{SCOT\_2005\_abund\_animation.pdf}. However, here we manually
appended them by incorporating the following code into the associated
.tex file and recompiling (see
\texttt{Appendix\_9\_\_seasonal\_dynamics\_animation\_.tex}).

\begin{Shaded}
\begin{Highlighting}[]
\CommentTok{# \textbackslash{}begin\{figure\}}
\CommentTok{# \textbackslash{}begin\{center\}}
\CommentTok{# \textbackslash{}animategraphics[controls, width=1\textbackslash{}linewidth]\{2\}\{occup_anim\}\{\}\{\}}
\CommentTok{# \textbackslash{}end\{center\}}
\CommentTok{# \textbackslash{}end\{figure\}}

\CommentTok{# \textbackslash{}newpage}

\CommentTok{# \textbackslash{}begin\{figure\}}
\CommentTok{# \textbackslash{}begin\{center\}}
\CommentTok{# \textbackslash{}animategraphics[controls, width=1\textbackslash{}linewidth]\{2\}\{abund_anim\}\{\}\{\}}
\CommentTok{# \textbackslash{}end\{center\}}
\CommentTok{# \textbackslash{}end\{figure\}}
\end{Highlighting}
\end{Shaded}

\begin{figure}
\begin{center}
\animategraphics[controls, width=1\linewidth]{2}{occup_anim}{}{}
\end{center}
\end{figure}

\newpage

\begin{figure}
\begin{center}
\animategraphics[controls, width=1\linewidth]{2}{abund_anim}{}{}
\end{center}
\end{figure}

\end{document}
